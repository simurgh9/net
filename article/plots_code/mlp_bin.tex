\documentclass[tikz]{standalone}
\begin{document}

\def\layersep{2.5cm}
\newcommand*{\eq}{=}

\begin{tikzpicture}[shorten >=1pt,->,draw=black!50, node distance=\layersep]
    \tikzstyle{every pin edge}=[<-,shorten <=1pt]
    \tikzstyle{neuron}=[circle,fill=black!25,minimum size=10pt,inner sep=0pt]
    \tikzstyle{input neuron}=[neuron, fill=green!50];
    \tikzstyle{output neuron}=[neuron, fill=red!50];
    \tikzstyle{hidden neuron}=[neuron, fill=blue!50];
    \tikzstyle{annot} = [text width=4em, text centered]

    % Draw the input layer nodes
    \foreach \name / \y in {1,...,10}
    % This is the same as writing \foreach \name / \y in {1/1,2/2,3/3,4/4}
        \node[input neuron, pin=left:$\scriptscriptstyle $\tiny bit$_{\y}$] (I-\name) at (0,-0.6*\y) {$\scriptscriptstyle a^{(1)}_{\y}$};

    % Draw the hidden layer nodes
    \foreach \name / \y in {1,...,3}
        \path[yshift=0.5cm]
            node[hidden neuron] (H-\name) at (\layersep,-\name-1.5) {$\scriptscriptstyle a^{(2)}_{\y}$};

    % Draw the output layer node
    \foreach \name / \y in {1,...,2}
        \path[yshift=0.5cm]        
            node[output neuron] (O-\name) at (2*\layersep,-\y-2) {$\scriptscriptstyle a^{(3)}_{\y}$};

    % Connect every node in the input layer with every node in the
    % hidden layer.
    \foreach \source in {1,...,10}
        \foreach \dest in {1,...,3}
            \path (I-\source) edge (H-\dest);

    % Connect every node in the hidden layer with the output layer
    \foreach \source in {1,...,3}
        \foreach \dest in {1,...,2}
            \path (H-\source) edge (O-\dest);

    % Annotate the layers
    \node[annot,above of=H-2, node distance=2.7cm] (hl) {\tiny Hidden layer \\ $\vec{a}^{(L-1)}$};
    % \node[annot,left of=hl] {Input layer \\ $\vec{a}^{(l)}$};
    \node[annot,right of=hl] {\tiny Output layer \\ $\vec{a}^{(L)}$};
\end{tikzpicture}
% End of code
\end{document}